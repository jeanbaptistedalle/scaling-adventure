\documentclass[a4paper,10pt]{article}

\usepackage[utf8]{inputenc}
\usepackage[T1]{fontenc}
\usepackage{lmodern}
\usepackage{geometry}
\geometry{letterpaper}

\usepackage{doc}
\usepackage{url}

\usepackage{graphicx}
\usepackage{epstopdf}
\DeclareGraphicsRule{.tif}{png}{.png}{`convert #1 `dirname #1`/`basename #1 .tif`.png}

\title{Protocole}
\author{BDGH}
\date{}

\begin{document}

\maketitle
\abstract{ Nous allons décrire dans la suite de ce document le protocole mis en place au sein de l'application de dessin collaboratif. Le protocole à été mis en place afin de permettre la connexion et la déconnexion d'un utilisateur, la transmission du dessin en temps réel entre tous les utilisateurs de l'application, la prise de contrôle sur le dessin en cours d'un utilisateur, etc... }
\begin{multicols}{2}

\section{Format des messages échangés.}
Afin de communiquer entre les clients et le serveurs nous avons définis un format précis de messages à envoyer qui permettra de savoir qui à envoyé le message d'une part et de faire passer une commande / une instructions et éventuellement (suivant le type de la commande) des paramètres ou des informations complémentaires.

Les messages sont donc constitués de trois parties : 
\begin{itemize}
\item[1] L'adresse du destinataire, sur 4 octets.
\item[2] Le type de la commande, sur 4 octets également.
\item[3] La taille du contenu du message stocké sur 4 octets.
\item[4] Enfin, le contenu du message.
\end{itemize}

\textit{Du point de programmation les messages sont stocké dans des byte[] et non pas de char[], car en Java un char est stocké sur 2 octets au lieu de 1. Pour des raisons de compatibilité avec d'autres langages nous avons donc pris la précaution de stocker les caractères des messages transmis sur des octets.}

Le format ainsi construit des message nous permet donc aisément de savoir qui à envoyé le message et donc de lui répondre, d'envoyé des informations au client ou au serveur. Cela permet aussi au serveur d'envoyer des ordres aux clients (Le serveur peut indiquer à tout les clients que c'est au tour de tel client de prendre le contrôle du dessin), ainsi qu'au client d'envoyer des demandes au serveur (Le client peut demander à prendre la main par exemple).

\section{Types de messages échangés.}
\end{multicols}
\end{document}